%%% Template by Mikhail Klassen, April 2013
%%% 
\documentclass[11pt,letterpaper]{article}
\newcommand{\workingDate}{\textsc{2019 $|$ April $|$ 2}}
\newcommand{\userName}{Jordan Sturtz}
\newcommand{\institution}{ Oregon State University}
\usepackage{researchdiary_png}
\usepackage{listings}
% To add your univeristy logo to the upper right, simply
% upload a file named "logo.png" using the files menu above.

\begin{document} \univlogo

\title{CS 475 - Parallel Programming}
{\Huge Project 0 Writeup}\\[5mm]

\begin{enumerate}
  \item \textbf{Machine} 
      
      I ran this program on the flip3 engr server. 

  \item \textbf{Performance} 

    I had very inconsistent results for each execution of the code for 
      small values of ARRAYSIZE (1000 - 10000). Only after I tried the code with arrays of
      length 100 million did it produce consisent results. Moreover, I noticed
      that my speedup never exceeded 2, which was suspicious. I did some
      digging and found that I did not have four cores available, so it was
      not actually delegating four threads for the code. After I ran it on
      the school server, I had much better results for the speedup. 

  \item \textbf{Times} 

    For my code, I created three integer arrays each with 10,000,000 items. 
    I ran the benchmarks 100 times and kept only the peak performance for each. 
    I also calculated the average performance to ensure that my result was 
    not an outlier. The output of my code is below: 
    \begin{lstlisting}
      Using 1 threads
      Peak Performance =  2183.41 Mega Mults/Sec
      Avg Performance =  2032.73 Mega Mults/Sec
      Using 4 threads
      Peak Performance =  7020.86 Mega Mults/Sec
      Avg Performance =  6615.97 Mega Mults/Sec
    \end{lstlisting}

  \item \textbf{Speedup}

    The above unit of measurement is in terms of how many times the 
    parallelized part can be executed in 1000000 seconds: 
    Mega Mults / Sec =  ARRAYSIZE / (1000000 * (time1 - time0)). 
    Since the speedup is equal to the time for single threaded run divided by
    the time for the nth threaded run, we can calculate this result by 
    dividing Mega Mults for the four-threaded divided by the Mega Mults for
    the single threaded: 
    \[ \frac{7020.86}{2183.41} = \mathbf{3.22} \]

    My guess is that the reason the speedup is less than 4 is that there is 
    some time lost in delegating the data of the for loop to four separate 
    threads and then recombining the results. Moreover, there is some error
    in measurement from executing one program after another, so we would 
    similarly expect some deviation from the theoretical speedup of 4.

  \item \textbf{Parallel Fraction:}

$ (\frac{4}{3}) * (1 - (\frac{1}{3.22})) = \mathbf{0.92}$

\end{enumerate}
\end{document}
