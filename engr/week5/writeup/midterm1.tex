%%% Template by Mikhail Klassen, April 2013
%%% 
\documentclass[11pt,letterpaper]{article}
\newcommand{\workingDate}{\textsc{2019 $|$ April $|$ 2}}
\newcommand{\userName}{Jordan Sturtz}
\newcommand{\institution}{ Oregon State University} \usepackage{researchdiary_png}
\usepackage{listings, lstautogobble}
\usepackage{graphicx}
\usepackage{wrapfig}
\usepackage{enumitem}

\lstset{
  autogobble=true, 
  breaklines=true
}

% To add your univeristy logo to the upper right, simply
% upload a file named "logo.png" using the files menu above.

\begin{document} \univlogo

\title{CS 362 - Midterm}
{\Huge CS 362 - Midterm, Question 1}\\[5mm]
\begin{enumerate}[label=\arabic*.]
  \item \textbf{Summary}

    My overall testing strategy has several components. First, I would do
    some manual testing to try to cover a particular equivalence partition
    of possible inputs. The goal in this case is to make sure that the
    program gets the right output for a select few important cases.
    This will involve positive, negative, and boundary cases enumerated
    below. 
    Second, because I am concerned about how the code will handle unusual
    characters, I would design a random tester. This tester will be 
    guaranteed to produce positive test cases, and so we can test whether
    the program is correctly handling all possible characters that might
    be stored in the files. Third, I would use coverage software to
    improve the coverage of these tests. 

  \item \textbf{Manual Testing}

    The positive cases for my manual testing will be those for which either
    something or nothing is outputted to the screen. I will be treating
    the "negative" cases as those for which the program should produce
    some kind of error. So, for example, if a string is not found in a file,
    I will be treating that as a positive case where nothing is outputted.
    But if the program is given an invalid number of arguments, that will
    be considered a negative case where an error should be outputted. Since
    the assignment description was not specific about this, I will assume
    the program will output to stderr some specific error. 

    The positive cases include cases where the string in in.txt is  
    only partially found in out.txt, the case where there are multiple 
    instances of the in.txt string found in out.txt, and the case where
    the match is case-insensitive but not case sensitive.
    
    The negative cases include those where the number of arguments are incorrect, when the
    arguments do not specify files that exist, or when the arguments specify files that
    do exist but the permissions are not available for reading. 

    The border cases include the case where the string in in.txt matches
    a string in out.txt except that in.txt has an extra newline character
    or carriage return character and the cases where the input or output
    files are empty. 

    \begin{itemize}
      \item \textbf{Positive Cases}
        \begin{itemize}

          \item Does the program correctly output nothing when the string in
            in.txt matches a string in out.txt in a case insensitive way but
            not in a case sensitive way?

            As an example, I would check that nothing is outputted to stdout
            if in.txt held cs362 and out.txt held "Welcome to CS362 Software
            Engineering II." 

        \item Does the program correctly output multiple strings to stdout 
          when the same string from in.txt is found on multiple lines in 
          out.txt?
          
            I would write a program that writes a string to in.txt
            and then embeds that string as a substring of a longer string. I would
            do this multiple times and then assert that the output redirected from
            stdout matches the multiple strings I have stored in the testing program.

      \item Does the program correctly output a line only once even if it has duplicates
        of the same string found in in.txt? For example, suppose that in.txt has the 
            contents "Hello" and out.txt has only the contents "Hello Hello\\n" on one line. In this case, 
            the line should be outputted once, not twice. 

            To test this, I would run the program with the above example contents and
            check that only one line was outputted to stdout. 

        \end{itemize}

      \item \textbf{Negative Cases}

        \begin{itemize}
          \item Does the program output an error to stderr if the program is
            called with a number of arguments other than two?

            I would write a script that runs ./paraphrase with 
            zero arguments, one argument, three arguments, and 
            1000 arguments, and then checks that in each case, 
            a message was written to stderr and nothing to stdout. 

          \item Does the program output an error to stderr if the program is
            called where the either the first argument or second argument
            does not exist? 

            I would create a script that runs ./paraphrase with an argument naming a file 
            that does not exist, then checks that a message was written to stderr and
            nothing to stdout. I would do this for both argument 1 and 2.

          \item Does the program output an error to stderr if the program is
            called where the second argument filename that exists but without
            the permissions for reading? 

            I would write a script that creates both in.txt and out.txt but 
            without proper read permissions, and then run the script to see if it
            outputs an error to stderr. Since the default behavior of such scripts
            is likely to send an error to stderr if a file attempts to be opened
            without proper permissions, whether the code correctly catches this
            bug depends on the details of how the script should be designed. 
            For example, it might not want to output a specific error and just 
            output whatever error would normally be thrown by opening a file
            without proper permissions. 


        \end{itemize} 
      
      \item \textbf{Boundary Cases}

        \item Does the program correctly output nothing to stdout when
          the only difference between the contents of in.txt contain
          a newline character or carriage return which does not appear in
          out.txt?

              To test this, I would create two near-identical strings, one
              to be written to in.txt and one to out.txt. The one to be
              written to in.txt would be one that is the same as the one
              to be written to out.txt except that the in.txt string will
              be concatenated with one newline character. I would repeat
              this process for the carriage return. This case is a kind of
              boundary case to ensure that it treats newline characters
              as themselves characters that must exist in out.txt 
              for the line to be outputted to stdout. 

        \item Does the program correctly output 100,000 line to stdout
          when the string in in.txt is found on those same 100000 lines in
          out.txt? (or some other arbitrarily large value)

            The purpose of this boundary case is 
            just to see how it handles cases where the number of 
            lines is very large. 

          \item Does the program correctly output 1 line to stdout when the
            string is found on the last line of the code? 

            To test this, I would write a program that fills some arbitrary
            number of lines with garbage, and then fills the last line 
            with a string that contains the substring written to in.txt. 

          \item Does the program correctly output 1 line to stdout when
            the string is found on the first line of the code? 

            To test this, I would write a program that fills some arbitrary
            number of lines with garbage, and then fills the first line 
            with a string that contains the substring written to in.txt. 

    \end{itemize}

  \item \textbf{Random Testing}

    There are many interesting cases that might be missed in manual testing. Two 
    obvious ones to test would be (1) whether the program can handle cases where
    the character data is unusual and (2) whether it consistently gets the right
    number of lines outputted to stdout. 

    For the first case, I would design my random tester to produce a random string
    to be written to in.txt. I would then write a function called embedString that
    would embed that string as a substring of different randomly generated string and
    append that new string to out.txt along with a newline character. I would
    store this randomly generated string and check that the redirected output from
    stdout matches character for character. 
    
    For the second case, I would modify the above approach slightly. Because we are interested
    in counting the number of lines that get outputted, I would create a second string
    to be appended to out.txt. I would check that this second string does not contain
    the substring written to in.txt. Therefore, there would be two possible string that 
    could be appended to out.txt. I would randomly generate an integer representing the
    number of outputs we would expect, and I would write that many lines of the randomly generated
    string to out.txt. I would then generate another random integer and write the string
    that doesn't contain the in.txt string. Since I would know how many lines contain the
    substring, I would check that it outputs the same number of lines to stdout. I would
    iterate this millions of times to check that it is working for arbitrary sizes of 
    lines. 
    
  \item \textbf{Coverage}
    
    Assuming this is white-box testing, I would use some kind of coverage software in
    both manual and random testing to ensure close to 100\% branch and statement 
    coverage. I would then use the results from this percent coverage to modify my tests
    so that (1) in the manual cases, I have 100\% coverage and (2) in the random case, 
    I increase the percentage of those branches that are rarely taken. 
    Based on my interpretation of how the source code would be written, I would
    guess that there aren't any issues involving path coverage. 

\end{enumerate}
\end{document}
