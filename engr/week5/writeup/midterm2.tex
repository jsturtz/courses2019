%%% Template by Mikhail Klassen, April 2013
%%% 
\documentclass[11pt,letterpaper]{article}
\newcommand{\workingDate}{\textsc{2019 $|$ April $|$ 2}}
\newcommand{\userName}{Jordan Sturtz}
\newcommand{\institution}{ Oregon State University} \usepackage{researchdiary_png}
\usepackage{listings, lstautogobble}
\usepackage{graphicx}
\usepackage{wrapfig}
\usepackage{enumitem}

\lstset{
  autogobble=true, 
  breaklines=true
}

% To add your univeristy logo to the upper right, simply
% upload a file named "logo.png" using the files menu above.

\begin{document} \univlogo

\title{CS 362 - Midterm}
{\Huge CS 362 - Midterm, Question 2}\\[5mm]
The three cases I chose should cover most if not all the branches
of this function. 
Case 1 is the case where n does not exist
in container C. This case is really testing that the function does not delete anything
in the container after calling removeAll with a value that doesn't exist in
C. The only way to test this with just the provided signatures is to assert that
size hasn't changed. Case 2 is that n exists some known k number of times in C. 
I will run this case with one 
negative value, one positive value, and zero to ensure the same behavior for
negatives, positives, and zero. For all three values, I use both size and get to test
correctness. The final case is when C is completely empty. This boundary case is
testing whether the function does nothing and throws no error. 
\begin{enumerate}
  \item Test Case 1: n does not exist in C

    The below code will create a container that does not contain c. 
    It will then copy that container, call removeAll, and then
    check that the size has not changed.
  \begin{lstlisting}
  
  struct container C, testC;
  &C = newContainer();
  &testC = newContainer();
  
  // populate C with values less than 100
  for (int i = 0; i < 100; i++) 
  {
    add(i, &C);
  }
  
  // copy memory to testC
  memcpy(&C, &testC, sizeof(struct container C));
  
  // run function on testC
  removeAll(101, &testC);
  
  // check that size hasn't changed, i.e. nothing removed
  assert(size(&C) == size(&testC));

  \end{lstlisting}

  \item Test Case 2: n exists in C k times

    The below code will create a container that contains n. It will then
    copy that container to another, call removeAll, and then check
    that the size has decreased by the number of times, k, that n was
    added to c. I will also called get(\&C, n) and assert that it returns
    false. I will do this three times, one for positive k, one for 
    negative k, and one for 0. 

  \begin{lstlisting}
  
  struct container C, blankC;
  &C = newContainer();
  &blankC = newContainer();
  
  // populate C with values less than |100|
  for (int i = 0; i < 100; i++) 
  {
    add(i, &C);
    add(-i, &C);
  }

  memcpy(&blankC, &C, sizeof(struct container));

  for (int i = 0; i < 100; i++) 
  {
    add(300, &C);       // will check that number of instances of 300 == 100
  }
  
  removeAll(300, &C);
  assert(size(&C) == 100);
  assert(get(&C, 300) == 0);

  memcpy(&C, &blankC, sizeof(struct container));

  for (int i = 0; i < 100; i++) 
  {
    add(-300, &C);       // will check that number of instances of -300 == 100
  }

  removeAll(-300, &C);
  assert(size(&C) == 100);
  assert(get(&C, 300) == 0);

  memcpy(&C, &blankC, sizeof(struct container));

  for (int i = 0; i < 100; i++) 
  {
    add(0, &C);       // will check that number of instances of -300 == 100
  }

  removeAll(0, &C);
  assert(size(&C) == 100);
  assert(get(&C, 0) == 0);

  \end{lstlisting}

  \item Test Case 3: C is completely empty 

    The below code will create a container that has no values. 
    Then it will call removeAll and check that size is 0 
    afterwards. 
  \begin{lstlisting}
  
  struct container C, testC;
  &C = newContainer();
  
  // run function on testC
  removeAll(1, &C);
  
  // check that size hasn't changed, i.e. nothing removed
  assert(size(&testC) == 0);

  \end{lstlisting}
\end{enumerate}
\end{document}
