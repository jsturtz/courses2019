%%% Template by Mikhail Klassen, April 2013
%%% 
\documentclass[11pt,letterpaper]{article}
\newcommand{\workingDate}{\textsc{2019 $|$ April $|$ 2}}
\newcommand{\userName}{Jordan Sturtz}
\newcommand{\institution}{ Oregon State University} \usepackage{researchdiary_png}
\usepackage{listings, lstautogobble}
\usepackage{graphicx}
\usepackage{wrapfig}
\usepackage{enumitem}

\lstset{
  autogobble=true, 
  breaklines=true
}

% To add your univeristy logo to the upper right, simply
% upload a file named "logo.png" using the files menu above.

\begin{document} \univlogo

\title{CS 362 - Assignment 4}
{\Huge CS 362 - Assignment 4}\\[5mm]
\begin{enumerate}[label=\arabic*., leftmargin=*]
  \item \textbf{Random Testing}

    I wanted to randomize the variables that would seem to be most relevant to 
    bizzare undesirable behavior. Thus, I wrote a loop that on each iteration,
    will randomize the number of players (2 - 4), the decksize and contents for
    each player, the discard size and contents for each player, and the hand size
    and contents for each player. I made sure to randomize the deck and hand for
    each player such that they will never exceed MAX\_DECK and MAX\_HAND 
    defined in dominion.h. I also randomized the discard count such that it will
    be equal to MAX\_DECK - deckCount so that there cannot be any behavior due to
    the decksize increasing beyond MAX\_DECK due to a shuffle. 

    There are potential problems with this approach to setting random variables.
    For instance, it makes no attempt to stay consistent with the fact that
    each person's deck must come from fixed supply piles. However, the
    three functions I am testing do not depend on checking or modifying
    the supply piles, so I decided this issue was irrelevant. If any of the
    cards had a gain effect, it would be important to ensure a more game-realistic
    gamestate with respect to supply piles and player hands. 

    There is one other issue with adventurer that required a small modification. 
    Since the adventurer draws until two treasures are found, it is possible
    that randomly assigning the deck will not produce any treasures. To
    account for this, I added two random treasures to each player's deck in
    testing adventurerEffect.

    Once these random variables are assigned, each iteration calls its respective
    function, then runs a series of tests detailed below: 

    \begin{itemize}[leftmargin=*]

      \item \textbf{Smithy Card}
        \begin{itemize}[leftmargin=*]
          \item File: randomcardtest1.c
          \item Function to test: smithyEffect
          \item Tests
            \begin{itemize}[leftmargin=*]
              \item Number cards in hand increases by two
                \begin{lstlisting}
                ASSERT_TRUE(testG.handCount[player] == G.handCount[player] + 2);
                \end{lstlisting}

              \item Smithy added to playedCard array
                \begin{lstlisting}
                ASSERT_TRUE(testG.playedCards[testG.playedCardCount-1] == smithy);
                \end{lstlisting}

              \item Original hand remains unchanged

                \begin{lstlisting}
                ASSERT_TRUE(memcmp(testG.hand[player], G.hand[player], G.handCount[player] - 1) == 0);
                \end{lstlisting}

              \item Total card count should not have changed

                \begin{lstlisting}
                ASSERT_TRUE(G.handCount[player] + G.discardCount[player] + G.deckCount[player] + G.playedCardCount == testG.handCount[player] + testG.discardCount[player] + testG.deckCount[player] + testG.playedCardCount);
                \end{lstlisting}

              \item Top X of player hand == top X of deck before function call, where X is either top three of deck or remaining X from deck

                \begin{lstlisting}
                int topDeck[3];
                int topHand[3];
                memset(topDeck, '\0', 3);
                memset(topHand, '\0', 3);

                int numSet;
                if (G.deckCount[player] >= 3)
                  numSet = 3;
                else
                  numSet = G.deckCount[player];

                memcpy(topDeck, &G.deck[player][G.deckCount[player] - numSet], sizeof(int) * numSet);
                memcpy(topHand, &testG.hand[player][testG.handCount[player] - numSet], sizeof(int) * numSet);
                ASSERT_TRUE(isSameSet(topDeck, topHand, numSet, isGreaterThan));
                \end{lstlisting}

              \item Gamestate has not changed for any non-player variable

                \begin{lstlisting}
                ASSERT_TRUE(G.numPlayers == testG.numPlayers);
                ASSERT_TRUE(memcmp(G.supplyCount, testG.supplyCount, (treasure_map+1) * sizeof(int)) == 0);
                ASSERT_TRUE(memcmp(G.embargoTokens, testG.embargoTokens, (treasure_map+1) * sizeof(int)) == 0);
                ASSERT_TRUE(G.outpostPlayed == testG.outpostPlayed);
                ASSERT_TRUE(G.outpostTurn == testG.outpostTurn);
                ASSERT_TRUE(G.whoseTurn == testG.whoseTurn);
                ASSERT_TRUE(G.phase == testG.phase); 
                ASSERT_TRUE(G.numActions == testG.numActions); 
                ASSERT_TRUE(G.coins == testG.coins); 
                ASSERT_TRUE(G.numBuys == testG.numBuys); 
                \end{lstlisting}

              \item Gamestate has not changed for any other player

                \begin{lstlisting}
                for (j = 0; j < numPlayers; j++) {
                  if (j != player) {
                    ASSERT_TRUE(memcmp(G.hand[j], testG.hand[j], sizeof(int) * (G.handCount[j] - 1)) == 0);
                    ASSERT_TRUE(memcmp(G.discard[j], testG.discard[j], sizeof(int) * (G.discardCount[j] - 1)) == 0);
                    ASSERT_TRUE(memcmp(G.deck[j], testG.deck[j], sizeof(int) * (G.deckCount[j] - 1)) == 0);
                  }
                }
                \end{lstlisting}
          \end{itemize}
      \end{itemize}


      \item \textbf{Council Room Card}
        \begin{itemize}[leftmargin=*]
          \item File: randomcardtest2.c
          \item Function to test: councilRoomEffect
          \item Tests
        \begin{itemize}[leftmargin=*]
              \item Number cards in hand increases by three
                  \begin{lstlisting}
                  ASSERT_TRUE(testG.handCount[player] == G.handCount[player] + 3);
                  \end{lstlisting}
              \item Council Room added to playedCard array

                  \begin{lstlisting}
                  ASSERT_TRUE(testG.playedCards[testG.playedCardCount-1] == council_room);
                  \end{lstlisting}
              \item Original hand remains unchanged

                  \begin{lstlisting}
                  ASSERT_TRUE(memcmp(testG.hand[player], G.hand[player], G.handCount[player] - 1) == 0);
                  \end{lstlisting}
              \item Total card count should not have changed

                  \begin{lstlisting}
                  ASSERT_TRUE(G.handCount[player] + G.discardCount[player] + G.deckCount[player] + G.playedCardCount == testG.handCount[player] + testG.discardCount[player] + testG.deckCount[player] + testG.playedCardCount);
                  \end{lstlisting}

              \item Top X of player hand == top X of deck before function call, where X is either top four of deck or remaining X from deck
                  \begin{lstlisting}
                  int topDeck[4];
                  int topHand[4];
                  memset(topDeck, '\0', 4);
                  memset(topHand, '\0', 4);

                  int numSet;
                  if (G.deckCount[player] >= 4)
                    numSet = 4;
                  else
                    numSet = G.deckCount[player];

                  memcpy(topDeck, &G.deck[player][G.deckCount[player] - numSet], sizeof(int) * numSet);
                  memcpy(topHand, &testG.hand[player][testG.handCount[player] - numSet], sizeof(int) * numSet);
                  ASSERT_TRUE(isSameSet(topDeck, topHand, numSet, isGreaterThan));
                  \end{lstlisting}

              \item Gamestate has not changed for any non-player variable except numBuys

                  \begin{lstlisting}
                  ASSERT_TRUE(G.numPlayers == testG.numPlayers);
                  ASSERT_TRUE(memcmp(G.supplyCount, testG.supplyCount, (treasure_map+1) * sizeof(int)) == 0);
                  ASSERT_TRUE(memcmp(G.embargoTokens, testG.embargoTokens, (treasure_map+1) * sizeof(int)) == 0);
                  ASSERT_TRUE(G.outpostPlayed == testG.outpostPlayed);
                  ASSERT_TRUE(G.outpostTurn == testG.outpostTurn);
                  ASSERT_TRUE(G.whoseTurn == testG.whoseTurn);
                  ASSERT_TRUE(G.phase == testG.phase); 
                  ASSERT_TRUE(G.numActions == testG.numActions); 
                  ASSERT_TRUE(G.coins == testG.coins); 
                  ASSERT_TRUE(G.numBuys + 1 == testG.numBuys); 
                  \end{lstlisting}

              \item Gamestate has not changed for any other player except one card draw
                  \begin{lstlisting}
                   for (j = 0; j < numPlayers; j++) {
                     if (j != player) {
                       ASSERT_TRUE(G.handCount[j] + 1 == testG.handCount[j]);
                   
                       // Assertions depend on whether deck must be shuffled for draw, i.e. deckCount == 0
                       if (G.deckCount[j] != 0) {
                         ASSERT_TRUE(G.deckCount[j] == testG.deckCount[j] + 1);
                         ASSERT_TRUE(G.discardCount[j] == testG.discardCount[j]);
                        }
                       else
                         ASSERT_TRUE(testG.discardCount[j] == 0);
                     }
                   }
                  \end{lstlisting}
          \end{itemize}
      \end{itemize}

      \item \textbf{Adventurer Card}
        \begin{itemize}[leftmargin=*]
          \item File: randomtestadventurer.c
          \item Function to test: adventurerEffect
          \item Tests
        \begin{itemize}[leftmargin=*]
              \item Number cards in hand increases by 1
                  \begin{lstlisting}
                  ASSERT_TRUE(testG.handCount[player] == G.handCount[player] + 1);
                  \end{lstlisting}
              \item Adventurer added to playedCard array

                  \begin{lstlisting}
                  assert_true(testg.playedcards[testg.playedcardcount-1] == adventurer);
                  \end{lstlisting}

              \item Two new cards are treasures
                  \begin{lstlisting}
                  ASSERT_TRUE(testG.hand[player][testG.handCount[player] - 1] <= gold && 
                              testG.hand[player][testG.handCount[player] - 1] >= copper);
                  ASSERT_TRUE(testG.hand[player][testG.handCount[player] - 2] <= gold && 
                              testG.hand[player][testG.handCount[player] - 2] >= copper);
                  \end{lstlisting}

              \item Original hand remains unchanged

                  \begin{lstlisting}
                  ASSERT_TRUE(memcmp(testG.hand[player], G.hand[player], G.handCount[player] - 1) == 0);
                  \end{lstlisting}
              \item Total card count should not have changed

                  \begin{lstlisting}
                  ASSERT_TRUE(G.handCount[player] + G.discardCount[player] + G.deckCount[player] + G.playedCardCount == testG.handCount[player] + testG.discardCount[player] + testG.deckCount[player] + testG.playedCardCount);
                  \end{lstlisting}

              \item Gamestate has not changed for any non-player variable

                  \begin{lstlisting}
                  ASSERT_TRUE(G.numPlayers == testG.numPlayers);
                  ASSERT_TRUE(memcmp(G.supplyCount, testG.supplyCount, (treasure_map+1) * sizeof(int)) == 0);
                  ASSERT_TRUE(memcmp(G.embargoTokens, testG.embargoTokens, (treasure_map+1) * sizeof(int)) == 0);
                  ASSERT_TRUE(G.outpostPlayed == testG.outpostPlayed);
                  ASSERT_TRUE(G.outpostTurn == testG.outpostTurn);
                  ASSERT_TRUE(G.whoseTurn == testG.whoseTurn);
                  ASSERT_TRUE(G.phase == testG.phase); 
                  ASSERT_TRUE(G.numActions == testG.numActions); 
                  ASSERT_TRUE(G.coins == testG.coins); 
                  ASSERT_TRUE(G.numBuys == testG.numBuys); 
                  \end{lstlisting}

              \item Gamestate has not changed for any other player
                  \begin{lstlisting}
                  for (j = 0; j < numPlayers; j++) {
                    if (j != player) {
                      ASSERT_TRUE(memcmp(G.hand[j], testG.hand[j], sizeof(int) * (G.handCount[j] - 1)) == 0);
                      ASSERT_TRUE(memcmp(G.discard[j], testG.discard[j], sizeof(int) * (G.discardCount[j] - 1)) == 0);
                      ASSERT_TRUE(memcmp(G.deck[j], testG.deck[j], sizeof(int) * (G.deckCount[j] - 1)) == 0);
                    }
                  }
                  \end{lstlisting}
          \end{itemize}
      \end{itemize}
    \end{itemize}

  \item \textbf{Code Coverage}
  
    With number of iterations set to 10, the coverage results were the following: 
    \begin{lstlisting}
    Function 'adventurerEffect'
    Lines executed:88.24% of 17
    Branches executed:100.00% of 12
    Taken at least once:91.67% of 12
    No calls

    Function 'smithyEffect'
    Lines executed:100.00% of 6
    Branches executed:100.00% of 2
    Taken at least once:100.00% of 2
    No calls

    Function 'councilRoomEffect'
    Lines executed:100.00% of 11
    Branches executed:100.00% of 6
    Taken at least once:100.00% of 6
    No calls
    \end{lstlisting}
  
    The 100\% branch and statement coverage for smithyEffect and councilRoomEffect are uninteresting
    because both of these functions are guaranteed to hit all branches and therefore all 
    statements for all executions. For this reason, I chose to run adventurerEffect
    with a larger input size than ten to measure how long it takes to achieve 100\% 
    branch and statement coverage. I compiled and executed the program with NUMTRIES = 1000, 
    and received the following result: 

    \begin{lstlisting}
    Function 'adventurerEffect'
    Lines executed:100.00% of 17
    Branches executed:100.00% of 12
    Taken at least once:100.00% of 12
    No calls
    \end{lstlisting}

    This executed in seconds. 

  \item \textbf{Random Versus Unit Testing}

    I designed my unit tests for all three of the above functions for assignment-2 such that
    it would be guaranteed to hit all branches. Thus, strictly speaking, the coverage is
    the same for both the manual and random testing. 

    The major difference between this week's random testing and assignment-2's manual testing
    is that random testing covers far more possibilites because of how it randomizes the hands, 
    decks, and discards. For example, suppose that there the code in adventurerEffect has a 
    bug such that it will discard \emph{any} adventurer cards in hand upon being played. My manual
    testing did not cover this case. There is no guarantee that this week's random testing
    would cover this case, but since the hands are randomized, the greater the iterations
    the greater the chance that this fault would be discovered. Thus, even though the
    types of tests covered are virtually identical for both manual and random testing, we
    would expect the random testing to be far more robust in finding bugs. 

\end{enumerate}
\end{document}
