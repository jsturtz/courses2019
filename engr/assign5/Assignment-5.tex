%%% Template by Mikhail Klassen, h
%%% 
\documentclass[11pt,letterpaper]{article}
\newcommand{\workingDate}{\textsc{2019 $|$ April $|$ 2}}
\newcommand{\userName}{Jordan Sturtz}
\newcommand{\institution}{ Oregon State University} \usepackage{researchdiary_png}
\usepackage{listings, lstautogobble}
\usepackage{graphicx}
\usepackage{wrapfig}
\usepackage{enumitem}

\lstset{
  autogobble=true, 
  breaklines=true
}

% To add your univeristy logo to the upper right, simply
% upload a file named "logo.png" using the files menu above.

\begin{document} \univlogo

\title{CS 362 - Assignment 5}
{\Huge Assignment 5}\\[5mm]
\begin{enumerate}[label=\Roman*.]
  \item \textbf{Bug Report}
    \begin{itemize}[label=]
      \item \textbf{Adventurer: Missing Parameter}

        \begin{itemize}[label=]
          \item \textbf{ID:} 1
          \item \textbf{Description:} adventurerCard has missing parameter
          \item \textbf{Steps to Reproduce:} N/A
          \item \textbf{Summary:}
            Any cardEffect function must move the card when played into
            the playedCard array, and therefore must have the hand position
            parameter. This bug causes two other bugs:
            (1) adventurer card was not added to playedCard array
            and (2) total handcount is one greater than it should be.
          \item \textbf{Severity:} High
          \item \textbf{Code:} N/A
        \end{itemize}

      \item \textbf{Adventurer: Failing to Discard All Cards}
        \begin{itemize}[label=]
          \item \textbf{ID:} 2
          \item \textbf{Description:} adventurerCard does not add all 
            non-treasures drawn to discard
          \item \textbf{Summary:} The adventurerCard fails to discard
            all non-treasures because of an incorrect loop counter. This
            bug is independent of the failure to add adventurer to playedCards.
          \item \textbf{Severity:} High
          \item \textbf{Steps to Reproduce:} 
            \begin{enumerate}
              \item Initialize variables needed for basic game (see below)
              \item Initialize new game with variables
              \item Add adventurer to player's hand
              \item Add copper, copper, estate, estate to top of deck
              \item Store old discardCount
              \item call adventurerCard
              \item Check that new discardCount == old discardCount + 2
            \end{enumerate}
          \item \textbf{Code:} 

                \begin{lstlisting}
                  // 1. initialize variables needed for basic game
                  int pass;                             
                  int seed = 1000;                      
                  int numPlayers = 2;                   
                  int player = 0;                       
                  struct gameState G;
                  int k[10] = {
                    adventurer, embargo, village, minion, mine, cutpurse, sea_hag, tribute, smithy, council_room }; 

                  // 2. initialize game
                  initializeGame(numPlayers, k, seed, &G);

                  // 3. add adventurer to player's hand
                  G.hand[player][G.handCount[player]] = adventurer;
                  G.handCount[player]++;

                  // 4. add copper, copper, estate, estate to deck
                  G.deck[player][G.deckCount[player] + 0] = copper;
                  G.deck[player][G.deckCount[player] + 1] = copper;
                  G.deck[player][G.deckCount[player] + 2] = estate;
                  G.deck[player][G.deckCount[player] + 3] = estate;
                  G.deckCount[player] = G.deckCount[player] + 4;

                  // 5. store current discardCount
                  int oldDiscardCount = G.discardCount[player];

                  // 6. call adventurerCard
                  adventurerCard(&G, player);

                  // 7. Check that new discardCount == old discardCount + 2
                  assert(G.discardCount[player] == oldDiscardCount + 2;
                \end{lstlisting}

        \end{itemize}

      \item \textbf{Smithy: Draws only 2 cards}
        \begin{itemize}[label=]
          \item \textbf{ID:} 3
          \item \textbf{Description:} smithyCard does draw three cards
            from deck
          \item \textbf{Summary:} This bug caused several tests to fail:
            \begin{lstlisting}
              FAILED: New handCount == 6, but oldCount == 5
              FAILED: Cards added to hand do not match top three cards of deck
              FAILED: Deck has not decreased by exactly three
            \end{lstlisting}
            We would expect all three to fail if the code was not properly 
            drawing three cards from the top of the deck.

          \item \textbf{Severity:} High
          \item \textbf{Steps to Reproduce:} 
            \begin{enumerate}
              \item Initialize variables needed for basic game (see below)
              \item Initialize new game with variables
              \item Add smithy to player's hand
              \item Add three arbitrary cards to top of deck
              \item Store old deckCount
              \item call smithyCard
              \item Check that new deckCount + 3 == old deckCount
            \end{enumerate}
          \item \textbf{Code:} 

                \begin{lstlisting}
                  // 1. initialize variables needed for basic game
                  int handPos;                             
                  int seed = 1000;                      
                  int numPlayers = 2;                   
                  int player = 0;                       
                  struct gameState G;
                  int k[10] = {
                    smithy, embargo, village, minion, mine, cutpurse, sea_hag, tribute, smithy, council_room }; 

                  // 2. initialize game
                  initializeGame(numPlayers, k, seed, &G);

                  // 3. add smithy to player's hand
                  handPos = G.handCount[player];
                  G.hand[player][handPos] = smithy;
                  G.handCount[player]++;

                  // 4. add three arbitrary cards to deck
                  G.deck[player][G.deckCount[player] + 0] = copper;
                  G.deck[player][G.deckCount[player] + 1] = silver;
                  G.deck[player][G.deckCount[player] + 2] = gold;
                  G.deckCount[player] = G.deckCount[player] + 3;

                  // 5. store current deckCount
                  int oldDeckCount = G.deckCount[player];

                 // 6. call smithyCard
                  smithyCard(&G, player, handPos);

                  // 7. Check that new discardCount == old discardCount + 2
                  assert(G.discardCount[player] == oldDiscardCount + 2;
                \end{lstlisting}
        \end{itemize}

      \item \textbf{Council Room: Drawing Bug}
        \begin{itemize}[label=]
          \item \textbf{ID:} 3
          \item \textbf{Description:} Sometimes, players whose turn is not active do not draw one card
          \item \textbf{Summary:} Bug went undetected by my unit tests because my unit tests only included
            up to two players, and this bug is only detectable when number of players are
            greater than two. 

          \item \textbf{Severity:} Medium
          \item \textbf{Steps to Reproduce:} 
            \begin{enumerate}
              \item Initialize variables needed for basic game (see below)
              \item Initialize new game with variables, including four players
              \item Add council room to player's hand
              \item Store old handCounts of all players in array
              \item call councilRoomCard
              \item loop over players' hands and check that all non-active players hands increased by 1
            \end{enumerate}
          \item \textbf{Code:} 

                \begin{lstlisting}
                  // 1. initialize variables needed for basic game
                  int handPos;                             
                  int seed = 1000;                      
                  int numPlayers = 4;                   
                  int active = 0;                       
                  struct gameState G;
                  int k[10] = {
                    council_room, embargo, village, minion, mine, cutpurse, sea_hag, tribute, smithy, council_room }; 

                  // 2. initialize game
                  initializeGame(numPlayers, k, seed, &G);

                  // 3. add council room to player's hand
                  handPos = G.handCount[player];
                  G.hand[player][handPos] = smithy;
                  G.handCount[player]++;

                  // 4. Store old handCounts of all players in array
                  int oldCounts[4];
                  for (int i = 0; i < 4; i++) 
                  {
                    oldCounts[i] = G.handCount[i];
                  }

                  // 5. call councilRoomCard
                  councilRoomCard(&G, active, handPos);

                  // 7. loop over player hands to check if non-players' hands increased by 1
                  for (int i = 1; i < 4; i++) 
                  {
                    assert(oldCounts[i] + 1 == G.handCount[i]);
                  }
                \end{lstlisting}
        \end{itemize}
    \end{itemize}

  \item \textbf{Test Report}
  \item \textbf{Debugging}
    \begin{enumerate}[label=]
      \item \textbf{Adventurer: Missing Parameter}

        I discovered this bug right away because my tests of my dominion.c
        immediately threw errors due to incorrect number of parameters. I
        ran my tests anyway by removing the handPos argument and inspected the failures. These two
        bugs seemed relevant: 
        \begin{lstlisting}
        FAILED: Adventurer card was not properly played, i.e. added to playedCard array
        FAILED: Handcount should have increased by 1, instead increased by 2
        \end{lstlisting}
        The handcount would be off if the adventurer card was never moved from hand
        to played cards, and obviously the same would be said for the first failure.
        After adding the code that adds adventurer effect to the 
        playedCards array, and running again, both tests passed:
        \begin{lstlisting}
        PASSED: Adventurer card was properly played, i.e. adventurer added to playedCard array
	PASSED: HandCount increased by 1 (2 treasures - 1 adventurer)
        \end{lstlisting}
        
      \item \textbf{Adventurer: Does not discard all non-treasures}

        This bug only showed up when there were cards drawn from the deck
        that must be discarded. I had already recognized the code responsible
        for discarding newly drawn cards:
        \begin{lstlisting}
        while(z - 1 > 0) {
            // discard all cards in play that have been drawn
            state->discard[currentPlayer][state->discardCount[currentPlayer]++] = temphand[z-1]; 

            // decrement z
            z = z-1;
        }
        \end{lstlisting}
        When I ran gdb I noticed that the loop should have 
        executed twice since the player had drawn a province and a duchy,
        but the loop only executed once. The while loop should have been 
        written as 
        \begin{lstlisting}
        while(z - 1 >= 0) {
          ...
        }
        \end{lstlisting}

      \item \textbf{Smithy: Draws only 2 cards}
        
        This cause of this bug is easy to identify. The code inside smithyCard
        is not complicated:

        \begin{lstlisting}
        void smithyCard(struct gameState *state, int currentPlayer, int handPos) {
          int i;

          //+3 Cards
          for (i = 1; i < 3; i++) {
              drawCard(currentPlayer, state);
          }

          //discard card from hand
          discardCard(handPos, currentPlayer, state, 0);
          return;
        }
        \end{lstlisting}

        The drawCard function needs to be called three times, but the while
        loop counter begins at 1, which means the for-loop will only loop
        twice. 

      \item \textbf{Council Room: Drawing failure for non-active players}

        I almost did not discover this bug because my test suite, when 
        properly refactored, does not reveal any failed tests. This is
        because I only tested a case where the game has two players. The
        bug inside the dominion.c code is this: 

        \begin{lstlisting}
          for (i = 0; i < state->numBuys; i++) {
              if ( i != currentPlayer ) {
                  drawCard(i, state);
              }
          }
        \end{lstlisting}

        The intention of this code is to loop over all the players in the
        game and call drawCard if those players are not the active player.
        However, the terminating condition in the for loop says 
        \emph{state-$>$numBuys} rather than \emph{state-$>$numPlayers}, and
        therefore there might be an error if numBuys is ever different from
        numPlayers. Since my test run only had two players and one buy, 
        numPlayers and numBuys are the same by the time this block of code
        executes (council room increments numBuys). Therefore, my test
        suite completely missed this case. The only reason I noticed this
        bug was that I had been inspecting the code to refactor my tests
        and I happened by chance to see the error. 

    \end{enumerate}
\end{enumerate}
\end{document}
