%%% Template by Mikhail Klassen, April 2013
%%% 
\documentclass[11pt,letterpaper]{article}
\newcommand{\workingDate}{\textsc{2019 $|$ April $|$ 2}}
\newcommand{\userName}{Jordan Sturtz}
\newcommand{\institution}{ Oregon State University} \usepackage{researchdiary_png}
\usepackage{listings, lstautogobble}
\usepackage{graphicx}
\usepackage{wrapfig}
\usepackage{enumitem}

\lstset{
  autogobble=true, 
  breaklines=true
}

% To add your univeristy logo to the upper right, simply
% upload a file named "logo.png" using the files menu above.

\begin{document} \univlogo

\title{CS 362 - Assignment 2}
{\Huge Assignment 2}\\[5mm]
\begin{enumerate}
  \item \textbf{Refactoring}

    The five cards for which I refactored were Adventurer, Smithy, 
    Council Room, Village, and Great Hall. 
    I refactored the code to move all the code within the cardEffect function
    to new functions (residing directly above cardEffect) called
    adventurerEffect, smithyEffect, councilRoomEffect, greatHallEffect, and 
    villageEffect. This refactoring was
    simple and involved passing in as arguments only three variables from the
    cardEffect function: (1) the pointer to the gameState state, (2) the integer 
    representing the currentPlayer, and (3) the integer representing the handPos. All other
    variables that the original code used were declared and initialized 
    within the functions themselves. Since each of these conditional switches
    are terminal points in the program, each switch case merely returns the
    output from calling these functions. 

  \item \textbf{Bugs}

    I introduced one bug each to adventurerEffect, smithyEffect, councilRoomEffect, and greatHallEffect. 

    \begin{itemize}[leftmargin=*]
      \item \textbf{Adventurer}
        I made a small alteration in the Adventurer code that has the effect of failing to discard the
        appropriate number of cards. For Adventurer, the player keeps drawing cards until two treasures
        have been drawn. Every revealed card is discarded except for those two treasures. In the original code, 
        this is handled with the following loop: 

        \begin{lstlisting}
          while(z-1>=0){
            state->discard[currentPlayer][state->discardCount[currentPlayer]++]=temphand[z-1];
            z=z-1;
          }
        \end{lstlisting}
        
        I made a subtle change in the while loop condition so that it will fail to discard
        all the newly revealed cards. It will instead retain one revealed non-treasure 
        card in the player's hand:

        \begin{lstlisting}[breaklines=true]
          while(z-1>0){
            state->discard[currentPlayer][state->discardCount[currentPlayer]++]=temphand[z-1];
            z=z-1;
          }
        \end{lstlisting}

      \item \textbf{Smithy}
        For the Smithy, I added a line of code that will increase the player's number of buys:
        
        \begin{lstlisting}[breaklines=true]
          state->numBuys++;
        \end{lstlisting}

      \item \textbf{Council Room}
        For the Council Room, I messed up the for loop so that it draws three cards instead of four:
        \begin{lstlisting}[breaklines=true][autogobble=true]
          for (i = 0; i < 3; i++)
            {
              drawCard(currentPlayer, state);
            }
        \end{lstlisting}

      \item \textbf{Great hall}
        For the Great Hall, I changed the trashflag from a 0 to a 1 in the discardCard function so
        that the Great Hall card will be trashed rather than discarded when the player uses this
        card:
        \begin{lstlisting}[breaklines=true][autogobble=true]
          discardCard(handPos, currentPlayer, state, 1);
        \end{lstlisting}
    \end{itemize}
\end{enumerate}
\end{document}
